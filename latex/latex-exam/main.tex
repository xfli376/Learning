\documentclass[answers]{exam}

\usepackage{xeCJK} 		% 写中文要用到
\usepackage{zhnumber} 		% 可以把题号变为中文
\usepackage{graphicx} 		% 插入图片
\usepackage{hyperref} 		% 插入链接
\usepackage{amsmath} 		% 数学符号
\usepackage{booktabs} 		% 表格样式

\pagestyle{headandfoot}
\firstpageheadrule
\firstpageheader{电子科技大学}{随机过程与排队论}{期末考试}
\runningheader{电子科技大学}
{随机过程与排队论}
{期末考试}
\runningheadrule
\firstpagefooter{}{第\thepage\ 页(共\numpages 页)}{}
\runningfooter{}{第\thepage\ 页(共\numpages 页)}{}

\reversemarginpar

\begin{document}
\begin{LARGE}
\noindent 考试科目:\underline{量子力学与统计物理} \
\end{LARGE} 
考试形式:\underline{一页纸开卷} \
考试时间:\underline{2022} 年 \underline{06} 月 \underline{24}日 \\ {\vspace*{0.3em}}
本卷由\underline{\hspace*{0.5em}三\hspace*{0.5em}}部分组成,共 \underline{\hspace*{0.5em}7\hspace*{0.5em}} 页。 考试时长:\underline{\hspace*{0.5em}120\hspace*{0.5em}} 分钟。\\ {\vspace*{0.3em}}
成绩构成比例:平时成绩 \underline{\hspace*{0.5em}20\hspace*{0.5em}} \%, 期中成绩 \underline{\hspace*{0.5em}20\hspace*{0.5em}} \%, 期末成绩 \underline{\hspace*{0.5em}60\hspace*{0.5em}} \%  \\ 
\begin{table}[htb]   
\begin{center}  
\renewcommand{\arraystretch}{1.5} 
%\caption{Beecy.}  
%\label{table:1} 
\begin{tabular}{|m{1.5cm}<{\centering}|m{1.0cm}<{\centering}|m{1.0cm}<{\centering}|m{1.0cm}<{\centering}|m{1.0cm}<{\centering}|m{1.0cm}<{\centering}|m{1.0cm}<{\centering}|m{1.0cm}<{\centering}|m{1.5cm}<{\centering}|}   
\hline   ~题号 & 一 & 二 & 三 & 四 & 五 & 六 & ~ &~总分  \\ 
\hline ~ & ~ & ~ & ~ & ~ & ~ & ~ & ~ & ~ \\    
   ~得分 & ~ & ~ & ~ & ~ & ~ & ~ & ~ & ~ \\   
  ~ & ~ & ~ & ~ & ~ & ~ & ~ & ~  & ~\\ 
\hline   
\end{tabular}   
\end{center}   
\end{table}

%\setlength{\marginparsep}{1.7cm}
%\putzdx %%装订线--奇页数
\section{\hspace{4cm} 一、填空题~(每空~1 分,共~30 分)}
\vspace{-2.1cm}

\begin{table}[htb]   
    \renewcommand{\arraystretch}{1.5} 
    \begin{tabular}{|m{2.5cm}<{\centering}|}   
    \hline  得~分 \\ 
    \hline ~ \\  ~  \\  ~ \\ 
    \hline   
    \end{tabular}     
\end{table}

\clearpage

\end{document}