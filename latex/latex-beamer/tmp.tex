~\\
\解~(1) 宽度为a的一维无限深势阱, 波函数为:
\[ \psi_n(x) = \begin{cases}
\sqrt{\frac{2}{a}} \sin (\frac{n\pi x}{a}), \quad (0<x<a) \\ 
0, \qquad \qquad \qquad (x<0, x>a)    \qquad   (\text{2 分})
\end{cases} \]
(2) 宽度为2a的一维无限深势阱, 波函数为:
\[ \Psi_n(x) = \begin{cases}
\sqrt{\frac{1}{a}} \sin (\frac{n\pi x}{2a}), \quad (0<x<2a) \\ 
0, \qquad \qquad \qquad (x<0, x>2a)  \qquad   (\text{2 分})
\end{cases} \]
(3) 处于第二激发态的概率:
\[ \begin{aligned}
    P_{31} &= \left\langle \Psi_3(x) | \psi_1(x) \right\rangle ^2  \qquad   (\text{2 分}) \\
    &= \left|\int _0 ^a \sin (\frac{3\pi x}{2a}) \sin (\frac{\pi x}{a}) dx \right|^2 \\
    &= \frac{32}{25\pi^2}\sin ^2 (\frac{5}{2}\pi) \\ 
    &= \frac{32}{25\pi^2}  \qquad   (\text{2 分}) 
\end{aligned}\] 

\解~(1) 自由粒子的势函数$U=0$,其哈密顿为
\[ \begin{aligned} H &= T+U = \frac{p^2}{2m} \\ 
    &=\frac{p_x^2 + p_y^2 + p_z^2 }{2m} = H(x) + H(y) 
    + H(z) \qquad (\text{1 分})
\end{aligned}\]
能量可表示为
\[ E = E_x +E_y +E_z  \]
(2)把 $H(x)$ 代入一维薛定谔方程
\[ \begin{aligned}
    H(x)\psi(x) &=E_x\psi(x) \\
    \frac{p_x^2}{2m}\psi(x) &= E_x\psi(x) \\
    - \frac{\hbar^2}{2m} \frac{d ^2 \psi(x)}{d x ^2 } &= E_x\psi(x) \qquad (\text{2 分}) \\
\end{aligned}\] 
解方程得:(式中取 $p_x = \sqrt{ 2m E_x}$)
\[ \psi(x) = \frac{1}{\sqrt{2\pi}}e^{-\frac{i}{\hbar}p_x x} \qquad (\text{2 分}) \]
三维薛定谔方程的解为:
\[ \psi(\vec{r}) =  (\frac{1}{2\pi})^\frac{3}{2} e^{-\frac{i}{\hbar}\vec{p}\cdot \vec{r}} \qquad (\text{1 分}) \]
(3)波函数为: 
\[ \psi(\vec{r},t) =  (\frac{1}{2\pi})^\frac{3}{2} e^{-\frac{i}{\hbar}(\vec{p}\cdot \vec{r}+Et)} \qquad (\text{2 分}) \]

\[ \begin{aligned}
    [V, p_x]\psi &= (Vp_x -p_x V)\psi \\
    &= Vp_x \psi -p_x V\psi  \qquad  (\text{ 2 分}) \\
    &= -i \hbar V \frac{\partial \psi}{\partial x } - (-i \hbar  \frac{\partial }{\partial x })(V\psi) \\ 
    &=  i\hbar\frac{\partial V }{\partial x }\psi
\end{aligned}\] 
因此, 有 $ [V, p_x] = i\hbar\frac{\partial V }{\partial x }  \qquad (\text{2 分})$