\documentclass[12pt,aspectratio=169]{beamer} 

\usepackage{hyperref} 
\usepackage{tikz}

\usepackage{tcolorbox}% 绘制彩色框.
\usepackage{multicol}%  one to muticolumn
\usepackage{ulem}% ulem: under line for emphasis.
\usepackage{tabularx}%  智能表格环境.
\RequirePackage{xhfill}% xhfill: 定制化线填充.
\usepackage{listings}%  for source code printing 

\usepackage{color}  % color content
\usepackage{xcolor}% \color{red!50!green!20!blue}
\usepackage{graphicx}
\usepackage{graphics}
\usepackage{subfigure}
%\graphicspath{{{images} {figs} { }}

\RequirePackage{comment}% comment: 提供注释环境

\RequirePackage{varioref}% varioref: 交叉引用.
\RequirePackage{cleveref}% cleveref: 交叉引用.
\RequirePackage[backend=biber,autolang=hyphen,style=gb7714-2015,gbalign=gb7714-2015,%style=gb7714-2015ms
gbstyle=false,url=false,doi=false,isbn=false,sorting=none]{biblatex}% biblatex: 参考文献.

\usepackage{xeCJK} %导入中文包
\RequirePackage{fontspec}% fontspec: 字体设置.
\usepackage{hyperref}

\RequirePackage{unicode-math}% unicode-math: opentype 数学字体.
\usefonttheme{professionalfonts}
\RequirePackage{fontawesome}% fontawesome: 提供可缩放矢量图标.
\setmainfont{Times New Roman}%  缺省英文字体.

\setCJKmainfont[ItalicFont={STKaitiSC-Regular}, BoldFont={STHeitiSC-Light}]{SimSong-Regular}% 衬线字体 缺省中文字体
\setCJKsansfont[BoldFont={STXihei}]{WeibeiSC-Bold} %sansfont 是无衬线字体,用于标题。
\setCJKmonofont{STFangsong}% 中文等宽字体
\setCJKfamilyfont{nicefont}{DFWaWaSC-W5}% 这里先把DFWaWaSC-W5导入CJK家族字体,后面用\CJKfamily{nicefont}调用。
\newcommand{\myfont}{\CJKfamily{nicefont}}%這一步是使用简化的\myfont
%\setCJKfamilyfont{niceft-S}{MLingWaiMedium-SC}% 
%\newcommand{\MLWti}{\CJKfontspec{niceft-S}}       %凌慧体-简, 用于手写体



%----------------------
%
% Presentation 环境
%
%---------------------
\mode<presentation>
%设置Beamer主题----- 效果见 https://hartwork.org/beamer-theme-matrix/
\usetheme{Antibes} %主题
%AnnArbor, Antibes, Bergen, Berkeley, Berlin, Boadilla, cambridgeUS, Copenhagen,
%Darmstadt, default,Dresden,Frankfurt,Goettingen,Hannover,Ilmenau,JuanLesPins,Luebeck,
%Madrid, Malmoe, Marburg, Montpellier,PaloAlto,Pittsburgh,Rochester,Singapore,Szeged,Warsaw
%------------------
\useoutertheme{smoothbars} %smoothbars, tree, sidebar, shadow, split, miniframes, infolines, default
\useinnertheme{rounded} % default ,circles ,rectangle, rounded
\usecolortheme{rose} % beaver,beetle, albatross,default, crane, dolphin, dove, mfly, lily, orchid,mrose, seagull, seahorse, whale, wolverine


%\setbeamertemplate{navigation symbols}{}%empty

% BackgrounB
\pgfdeclareimage[height=0.15\paperheight,width=0.15\paperwidth]{usetcxx}{usetcxx.png}
\pgfdeclareimage[height=\paperheight,width=\paperwidth]{myimage}{uestclogo.png}
\usebackgroundtemplate{\tikz\node[opacity=0.15,inner sep=0] {\pgfuseimage{myimage}};}

% 半透明化尚未出现的内容.
%\setbeamercovered{transparent}
% 幻灯标题字体设置.
\setbeamerfont{frametitle}{size=\large,series=\bfseries}
% 封页字体设置.
\setbeamerfont{title}{size=\Large,series=\bfseries}
\setbeamerfont{subtitle}{size=\footnotesize,series=\bfseries}
\setbeamerfont{author}{size=\normalsize}
\setbeamerfont{date}{size=\scriptsize}

\setbeamerfont{block title}{size=\normalsize}
\setbeamerfont{structure}{size=\normalsize,series=\bfseries}
% 脚注字号设置.
\setbeamerfont{footnote}{size=\scriptsize}
% 页眉页脚字体设置
\setbeamerfont{section in head/foot}{size=\scriptsize,series=\bfseries}
\setbeamerfont{subsection in head/foot}{size=\scriptsize,series=\bfseries}
\setbeamerfont{author in head/foot}{size=\scriptsize,series=\bfseries}
\setbeamerfont{institute in head/foot}{size=\scriptsize,series=\bfseries}
\setbeamerfont{date in head/foot}{size=\scriptsize,series=\bfseries}
\setbeamerfont{title in head/foot}{size=\scriptsize}
% itemize 环境序号设置.
\setbeamertemplate{itemize item}{\scriptsize\raise1.25pt\hbox{\textbullet}}
\setbeamertemplate{itemize subitem}{\scriptsize\raise1.25pt\hbox{\textbullet}}
\setbeamertemplate{itemize subsubitem}{\scriptsize\raise1.25pt\hbox{\textbullet}}
% 参考文献目录字号设置.
\renewcommand*{\bibfont}{\small}

% ----------------
% Headline Layout
% 格式: SECTION (ALL) | NAME OF SCU.
% ----------------
\defbeamertemplate{headline}{my headline theme}%
{%
	\leavevmode% 离开垂直模式
	\hbox{%
		\begin{beamercolorbox}[wd=.8\paperwidth,ht=5.75ex,dp=.25ex]{section in head/foot}%
			\hspace*{7em}%
			\usebeamerfont{section in head/foot}\insertsectionnavigationhorizontal{.6\textwidth}{}{}%
			\hspace*{1.175em}\normalsize\textbf{|}\vskip.02ex%
		\end{beamercolorbox}%
		\begin{beamercolorbox}[wd=.2\paperwidth,ht=5.75ex,dp=.25ex,right]{section in head/foot}%
			\pgfuseimage{uestcxx}\hspace*{2.6em}%
		\end{beamercolorbox}%
	}%
}
\setbeamertemplate{headline}[my headline theme]
% ----------------

\mode
<all>

\RequirePackage{comment}% comment: 提供注释环境
\usepackage{algorithm,algorithmic}
\usepackage{amsthm}
\usepackage{float}                           %HERE!!!!!!!!!!!!!!
%\newtheorem{proposition}{命题}
%\newtheorem{theorem}{定理}
%\newtheorem{definition}{定义}
%\newtheorem{example}{例}
\renewcommand\figurename{图}
\renewcommand\tablename{表}
\setbeamertemplate{theorems}[numbered]




 %   % 目录
 %   \AtBeginSection[]
 %  {
 %   \begin{frame}<beamer>
 %       \frametitle{\textbf{目录}}
 %       \tableofcontents[currentsection]
 %   \end{frame}
 %  }
 %	%%%%%%%%%%%%%%%%%%%%%%%%%%%%%%%%%
 % \beamerdefaultoverlayspecification{<+->}




%-------------------正文-------------------------%
%
%
\begin{document}  
%
%
%-----------------------------------------------%

%题目,作者,学校,日期                
\author {\LARGE 李小飞}
\title{\textbf{\Huge 量子力学与统计物理}}
\subtitle{Quantum mechanics and statistical physics}
\institute[电子科技大学]{{\large 光电科学与工程学院}}
\date{\today}

	%%%%%%%%%%%%%%%%%%%%%%%%%%%%%%%%%
    \frame[plain]{\titlepage}
    %%%%%%%%%%%%%%%%%%%%%%%%%%%%%%%%%
    \begin{frame}
        \frametitle{总目录}
        \tableofcontents
    \end{frame}
    %%%%%%%%%%%%%%%%%%%%%%%%%%%%%%%%%%

\section{一、基本操作}

\begin{frame}
    \frametitle{}
    第一部:分基本操作  
\end{frame}

\frame{\frametitle{本section目录}\tableofcontents[currentsection]}

\subsection{支持中文}

\begin{frame}
    \frametitle{1、设置中文环境}
    \begin{itemize}
     \item  导入中文包 \textbackslash usepackage\{ xeCJK \}
     \item  运行 xelatex
    \end{itemize}
\end{frame}

\begin{frame}
    \frametitle{2、设置中文字体}
    \begin{itemize}
     \item  {\myfont 中文字体 }
     \item  {\CJKfamily{nicefont} 中文字体}
    \end{itemize}
\end{frame}

\begin{frame}
    \frametitle{3、设置logo背景}

    完成

\end{frame}

\begin{frame}
    \frametitle{colorbox}
    \fcolorbox{gray}{yellow}{test}

\end{frame}


\subsection{公式}
\begin{frame}
    \frametitle{1、在线公式}

    \begin{itemize}
        \item<2-> 表格转换: Excel2\LaTeX~(\href{https://www.ctan.org/tex-archive/support/excel2latex/}{CTAN Excel2\LaTeX});
        \item<2-> 在线公式: \href{https://www.latexlive.com/}{LaTeX公式编辑器},~\href{https://mathpix.com/}{Mathpix}~ \href{https://mathf.itewqq.cn/}{图片在线转LaTeX}.
    \end{itemize}

\end{frame}

\begin{frame}[fragile]{添加线}
	\begin{multicols}{2}
		\verb|\uline|\hfill 下划线\qquad\uline{混}\\
		\verb|\uuline|\hfill 双下划线\qquad\uuline{元}\\
		\verb|\uwave|\hfill 波浪线\qquad\uwave{形}\\
		\verb|\sout|\hfill 删除线\qquad\sout{翼}\\
		\verb|\xout|\hfill 斜删除线\qquad\xout{太}\\
		\verb|\dashuline|\hfill 虚线\qquad\dashuline{极}\\
		\verb|\dotuline|\hfill 加点\qquad\dotuline{门}
	\end{multicols}
\end{frame}

\subsection{3.主题}

\begin{figure}
    \begin{minipage}[t]{0.5\linewidth}
    \centering
    \includegraphics[width=2.2in]{uestclogo.png}
    \caption{fig1}
    \label{fig:side:a}
    \end{minipage}%
    \begin{minipage}[t]{0.5\linewidth}
    \centering
    \includegraphics[width=2.2in]{uestclogo.png}
    \caption{fig2}
    \label{fig:side:b}
    \end{minipage}
\end{figure}

\subsection{4.主题}

\begin{frame}
	\frametitle{表}
	表格太麻烦了, 掌门说摸摸鱼, 编者觉得不错, 丢一个三线表示例. 当然也可以看看这个手册前面部分表格的源码.
	\begin{table}[htbp]
		\centering
		\caption{一些国风音乐}
		\label{tab:YixieGfyy}
		\begin{tabular}{rlc}
			作曲家 & 歌名 & 门中喜欢的友人 \\
			李志辉 & 小桥流水人家 & 门主 \\
			林海 & 无羁(器乐版) & 初号 \\
			吕秀龄 & 逆伦 & 小初 \\
			麦振鸿 & 从来只有一个人 & 编者(假的) \\
		\end{tabular}
	\end{table}
\end{frame}

\section{二、进阶操作}
\subsection{1.主题}

\begin{frame}[fragile]{代码环境演示}
	\begin{lstlisting}[language=c++]
        #include <iostream>
        int main()
        {
            std::cout << "Hello, World!" << std::endl;
        }  
    \end{lstlisting}
\end{frame}

\subsection{2.主题}

\begin{frame}[fragile,allowframebreaks]{数学环境}
	\begin{proof}{}
		请读者自证.
	\end{proof}
	\begin{definition}{马老卷}{MalaoJ}
		是混元形翼太极门的打砸工, 直系上峰是马凡王, 入门改姓马, 自称老卷, 实则不卷.
	\end{definition}
	\begin{lemma}{卷王森林法则}{JuanwangSlfz}
		源自未知高校学生, 此处略.
	\end{lemma}
	\begin{corollary}{狼人杀的重要性}{LangrenSdzyx}
		编者实习时听公司导师说面试有可能是趣味性游戏, 狼人杀感觉很符合, 所以玩狼人杀吧.
	\end{corollary}
\end{frame}


\begin{frame} [fragile,allowframebreaks]{数学环境}
\begin{algorithm}[H]                           % HERE!!!!!!!!!
\caption{Calculate $y = x^n$}          % give the algorithm a caption
\label{alg1}      % and a label for \ref{} commands later in the document
\begin{algorithmic}  % enter the algorithmic environment
\REQUIRE $n \geq 0 \vee x \neq 0$
\ENSURE $y = x^n$
\STATE $y \Leftarrow 1$
\IF{$n < 0$}
\STATE $X \Leftarrow 1 / x$
\STATE $N \Leftarrow -n$
\ELSE
\STATE $X \Leftarrow x$
\STATE $N \Leftarrow n$
\ENDIF
\WHILE{$N \neq 0$}
\IF{$N$ is even}
\STATE $X \Leftarrow X \times X$
\STATE $N \Leftarrow N / 2$
\ELSE[$N$ is odd]
\STATE $y \Leftarrow y \times X$
\STATE $N \Leftarrow N - 1$
\ENDIF
\ENDWHILE
\end{algorithmic}
\end{algorithm}
\end{frame}

\subsection{3.主题}

\begin{frame}
    \[ x^2+y^2=1 \]
    $$ x^2+y^2=1 $$, $ x^2+y^2=1 $, 
\end{frame}

\subsection{4.主题}

\section{二、高阶操作}
\subsection{1.主题}
\subsection{2.主题}
\subsection{3.主题}
\subsection{4.主题}

\end{document}