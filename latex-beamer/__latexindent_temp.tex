\documentclass[12pt,aspectratio=169,mathserif]{beamer} 

\usepackage{hyperref} 
\usepackage{tikz}
\usepackage{multirow}
\usepackage{color}      % color content
\usepackage{xcolor}
\usepackage{multicol}
\usepackage{ulem} % ulem: 添加线.

\usepackage{graphicx}
\usepackage{subfigure}

\usepackage{xeCJK} %导入中文包
\usepackage{hyperref}
\RequirePackage{unicode-math}% unicode-math: opentype 数学字体.
\RequirePackage{fontawesome}% fontawesome: 提供可缩放矢量图标.
\setmainfont{Times New Roman}%  缺省英文字体.
\setCJKmainfont[ItalicFont={STKaitiSC-Regular}, BoldFont={STHeitiSC-Light}]{SimSong-Regular}% 衬线字体 缺省中文字体
\setCJKsansfont[BoldFont={STXihei}]{WeibeiSC-Bold} %sansfont 是无衬线字体,用于标题。
\setCJKmonofont{STFangsong}% 中文等宽字体
\setCJKfamilyfont{nicefont}{DFWaWaSC-W5}% 这里先把DFWaWaSC-W5导入CJK家族字体,后面用\CJKfamily{nicefont}调用。
\newcommand{\myfont}{\CJKfamily{nicefont}}%這一步是使用简化的\myfont
\newcommand{\MLWti}{\CJKfontspec{MLingWaiMedium-SC}}       %凌慧体-简, 用于手写体

\mode<presentation>
%设置Beamer主题----- 效果见 https://hartwork.org/beamer-theme-matrix/
\usetheme{Antibes} %主题
%AnnArbor, Antibes, Bergen, Berkeley, Berlin, Boadilla, cambridgeUS, Copenhagen,
%Darmstadt, default,Dresden,Frankfurt,Goettingen,Hannover,Ilmenau,JuanLesPins,Luebeck,
%Madrid, Malmoe, Marburg, Montpellier,PaloAlto,Pittsburgh,Rochester,Singapore,Szeged,Warsaw
%------------------
\useoutertheme{smoothbars} %smoothbars, tree, sidebar, shadow, split, miniframes, infolines, default
\useinnertheme{rounded} % default ,circles ,rectangle, rounded
\usecolortheme{rose} % beaver,beetle, albatross,default, crane, dolphin, dove, mfly, lily, orchid,mrose, seagull, seahorse, whale, wolverine

%\setbeamertemplate{navigation symbols}{}%empty
\pgfdeclareimage[height=\paperheight,width=\paperwidth]{myimage}{uestclogo.png}
\usebackgroundtemplate{\tikz\node[opacity=0.15,inner sep=0] {\pgfuseimage{myimage}};}

% 半透明化尚未出现的内容.
%\setbeamercovered{transparent}
\mode
<all>


 %   % 目录
 %   \AtBeginSection[]
 %  {
 %   \begin{frame}<beamer>
 %       \frametitle{\textbf{目录}}
 %       \tableofcontents[currentsection]
 %   \end{frame}
 %  }
 %	%%%%%%%%%%%%%%%%%%%%%%%%%%%%%%%%%
 % \beamerdefaultoverlayspecification{<+->}




%-------------------正文-------------------------%
%
%
\begin{document}  
%
%
%-----------------------------------------------%

%题目,作者,学校,日期                
\author {\MLWti \LARGE 李小飞}
\title{\textbf{\Huge 量子力学与统计物理}}
\subtitle{Quantum mechanics and statistical physics}
\institute[电子科技大学]{{\large 光电科学与工程学院}}
\date{\today}

	%%%%%%%%%%%%%%%%%%%%%%%%%%%%%%%%%
    \frame[plain]{\titlepage}
    %%%%%%%%%%%%%%%%%%%%%%%%%%%%%%%%%
    \begin{frame}
        \frametitle{总目录}
        \tableofcontents
    \end{frame}
    %%%%%%%%%%%%%%%%%%%%%%%%%%%%%%%%%%

\section{一、基本操作}

\begin{frame}
    \frametitle{}
    第一部:分基本操作  
\end{frame}

\frame{\frametitle{本section目录}\tableofcontents[currentsection]}

\subsection{支持中文}

\begin{frame}
    \frametitle{1、设置中文环境}
    \begin{itemize}
     \item  导入中文包 \textbackslash usepackage\{ xeCJK \}
     \item  运行 xelatex
    \end{itemize}
\end{frame}

\begin{frame}
    \frametitle{2、设置中文字体}
    \begin{itemize}
     \item  {\myfont 中文字体 }
     \item  {\CJKfamily{nicefont} 中文字体}
    \end{itemize}
\end{frame}

\begin{frame}
    \frametitle{3、设置logo背景}

    完成

\end{frame}

\subsection{公式}
\begin{frame}
    \frametitle{1、在线公式}

    \begin{itemize}
        \item<2-> 表格转换: Excel2\LaTeX~(\href{https://www.ctan.org/tex-archive/support/excel2latex/}{CTAN Excel2\LaTeX});
        \item<2-> 在线公式: \href{https://www.latexlive.com/}{LaTeX公式编辑器},~\href{https://mathpix.com/}{Mathpix}~ \href{https://mathf.itewqq.cn/}{图片在线转LaTeX}.
    \end{itemize}

\end{frame}

\begin{frame}[fragile]{添加线}
	\begin{multicols}{2}
		\verb|\uline|\hfill 下划线\qquad\uline{混}\\
		\verb|\uuline|\hfill 双下划线\qquad\uuline{元}\\
		\verb|\uwave|\hfill 波浪线\qquad\uwave{形}\\
		\verb|\sout|\hfill 删除线\qquad\sout{翼}\\
		\verb|\xout|\hfill 斜删除线\qquad\xout{太}\\
		\verb|\dashuline|\hfill 虚线\qquad\dashuline{极}\\
		\verb|\dotuline|\hfill 加点\qquad\dotuline{门}
	\end{multicols}
\end{frame}

\subsection{3.主题}

\begin{frame}{图}
	\begin{figure}[h]
		\begin{subfigure}{.4\columnwidth}
			\centering
			\includegraphics%
			[width=.3\columnwidth]{uestclogo.jpeg}
		\end{subfigure}
		\quad
		\begin{subfigure}{.4\columnwidth}
			\centering
			\includegraphics%
			[width=.3\columnwidth]{uestclogo.jpeg}
		\end{subfigure}
		\caption{掌门常用的暂停}
	\end{figure}
	\begin{figure}[h]
		\begin{minipage}[t]{.4\columnwidth}
			\centering
			\includegraphics%
			[width=.3\columnwidth]{uestclogo.jpeg}
			\label{fig:ZhangmenBtdzt}
		\end{minipage}
		\quad
		\begin{minipage}[t]{.4\columnwidth}
			\centering
			\includegraphics%
			[width=.3\columnwidth]{uestclogo.jpeg}
			\label{fig:ZhangmenWsdzt}
		\end{minipage}
	\end{figure}
\end{frame}

\subsection{4.主题}

\section{二、进阶操作}
\subsection{1.主题}
\subsection{2.主题}
\subsection{3.主题}
\subsection{4.主题}

\section{二、高阶操作}
\subsection{1.主题}
\subsection{2.主题}
\subsection{3.主题}
\subsection{4.主题}

\end{document}